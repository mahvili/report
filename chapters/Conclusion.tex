\chapter{Conclusion}
\label{chapter:results}

\section{Conclusion}
This thesis investigated a new adaptive algorithmic approach using sparse grids and combination technique for solving interpolation types of problems. It is advantageous in terms of its robustness less floating point operations(FLOPs). The approach is also suitable for parallelization inherently separation of the problem to independant subproblems. Focused on reaching same order of grid points and memory use as conventional combination methods the approach tries to achieve better accuracy by faster adapting to region with higher order of error by starting from much coarser grid.\\

Several test functions as listed below verify that our implementation works as expected compared to high resolution full grid without adaptation and thus validated  to ensure the functionality under a variety of cases.
\begin{enumerate}
\item $f(x)=x^2+y^2$
\item $f(x)=x^2 \cdot y^2 $
\item $f(x)=\sqrt[2]{x} \cdot \sqrt[2]{y}$
\item $f(x)=16 \cdot x(1-x)y(1-y)$
\end{enumerate}
The default scheme of $\overrightarrow{l}=(4,4)$ for all the nodes of the tree including the root is assumed. In case the test function is not of hyperbolic form with respect to two dimension the bilinear interpolation used in combination technique gives the perfect solution. Simulations are run to observe the error for the local mesh refinement for two independent cases. The error accuracy is observed to be better than full grid mesh refinement. Test cases are also prepared in order to observe the mesh refinement when a predefined error function is enforced to the problem. Besides the predefined error, a simulation based error indicator for local grid has been investigated. The error accuracy is observed to be atleast as effective as the full grid mesh refinement. Under changing levels of refinement in a loop the logarithmic behavior of error as expected form the literature is observed for the solution suggested.\\
To examine the independent effect of the two key factors, threshold for the error and the level of refinement, two approaches are evaluated.
\begin{enumerate}
\item Threshold independent, refinement level changing.
\item Threshold changing, refinement level constant.
\end{enumerate}
The logarithmic error in both cases with an increase of that factor and thus level of refinement becomes saturated after one point since we there is no further refinement. Two different cases are different because one makes the height of the tree to be higher in some local part and the other one ensures that in each level we reach an expected local error. Test case 2 shows much higher error in the threshold-independent approach than the refinement-constant approach due to the definition of the function, in which error in concentrated in one subsection.\\
Overall, the conclusion is that the approach works as expected under different circumstances and while it has been validated for both adaptive and non-adaptive versions it shows the application of this method will be promising for future research and maybe a baseline for prospective scientist. Although It had worked for the cases presented but the author believes certain functions with higher derivatives or with singularities can bring some problems as one first need to detect the errors on the grid position. Therefore, If the variation of the function is inside the blocks basically this method will not work.

\section{Suggestions for future works}
As described earlier the interpolation problem is just a baseline for the adaptive mesh refinement combination technique. However, usually problems which has been tackled extensively in relevant literature e.g. regression and partial differential equation problem require higher complexity. This is becasue the numerical operations there are much more complex as they can be even non-linear in format. Thus definition of error indicator in such problems might be tricky but once we can achieve that we can investigate whether the current method can be applied there. 
