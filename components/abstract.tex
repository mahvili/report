% Abstract for the TUM report document
% Included by MAIN.TEX


\clearemptydoublepage
\phantomsection
\addcontentsline{toc}{chapter}{Abstract}	


\vspace*{2cm}
\begin{center}
{\Large \bf Abstract}
\end{center}
\vspace{1cm}

Finally some general important points and remarks will be discussed. These remarks will show the important factors at play in these kind of problems and maybe give insight for future work.
1. This new algorithmic concept is based on the independent solution of many problems with reduced size and their linear combination. \cite{Griebel1992}\\
2. The basic idea of combniaton technique is same as for multilevel splitting of finite element spaces is to replace the usual nodal bases of the finite element spaces by hierarchical bases.\cite{Yserentant1986} \\
3. nearly of the same optimal computational complexity as the conventional multigrid methods but which is free of many of its restrictions. \cite{Yserentant1986} \\
4. We think that one of the main advantages of our method is its robustness and the fact that its speed of convergence does not depend on the regularity properties of the considered boundary value problem or on a regular refinement.\cite{Yserentant1986} \\
5. The natural coarse grain parallelism of the combination method makes it prefectly suited for MIMD parallel computers and distributed processing on workstation networks.\cite{Griebel1992} \\

6. Consequently, one of the main advantages of the combination technique stems from the properties of sparse grids [1] In comparison to the standard full grid approach the number of grid points can be reduced significantly. Another advantage has to be seen in the simplicity of the combination concept its inherent parallel structure and its framework property allowing the integration of existing solvers for partial differential equations.\cite{Bungartz1994}\\

7. In elliptic problems the smoothness of the solution may be disturbed where the data is non-smooth. The form of the domain or the need for local refinement may make the use of uniform meshes difficult.  The local smoothness of the solution is a basic characteristic of many elliptic problems, so that extrapolation can be used locally, even when the global solution is non-smooth.\cite{Rude1994} \\

8. The simulation of complex real life experiments usually needs a great deal of computing time and returns vast amounts of data. Thus, in order to obtain sufficiently accurate simulation results, it is necessary to find algorithms which economize on both computing time and storage space.\cite{Griebel1995} \\

9. One way of using sparse grids efficiently involves hierarchical, tree-like data structures and special algorithms for both the discretization and the solution. Since conventional solvers usually do not provide means for dealing with hierarchical data structures, they cannot be employed for solving problems on sparse grids. Thus, new algorithms and new codes have to be developed in order to compute solutions on sparse grids efficiently 
