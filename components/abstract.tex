% Abstract for the TUM report document
% Included by MAIN.TEX


\clearemptydoublepage
\phantomsection
\addcontentsline{toc}{chapter}{Abstract}	


\vspace*{2cm}
\begin{center}
{\Large \bf Abstract}
\end{center}
\vspace{1cm}

This new algorithmic concept is based on the independent solution of many problems with reduced size and their linear combination using combination technique\cite{Griebel1992}. This technique has nearly same optimal computational complexity as the conventional multigrid methods. The main advantages of this method is its robustness and less memory consumption. The natural characteristic parallelism of the combination method makes it prefectly suited for MIMD parallel computers, distributed processing on workstation networks and multicore processors.\cite{Griebel1992}\\ 

One of the main advantages of the combination technique in comparison to the standard full grid is that it requires significantly less grid points. Another advantage is seen in the simplicity of the combination concept its characteristic parallel structure and its fundamental property allowing the integration of legacy solvers for partial differential equations.\cite{Bungartz1994}\\

One way of using sparse grids efficiently involves hierarchical, tree-like data structures and special algorithms for both the discretization and the solution. Since conventional solvers usually do not provide means for dealing with hierarchical data structures, they cannot be employed for solving problems on sparse grids. Thus, new algorithms and new codes have to be developed in order to compute solutions on sparse grids efficiently. The method that is implemented in this thesis employs the concept of non-linear propagation  in the data structure. For this purpose, the thesis focuses on quad trees and related tree traversal techniques.  The combination method that is used in the thesis aims to operate using fewer FLOPs per iteration than the conventional full grid approach. The combination technique used in this thesis differs from the other combination technique in a way that it projects all the sub-levels grids into a full grid and then combine them to find a solution. The problem of interpolation is also under investigation in the thesis as different interpolation can lead difference in accuracy of the result.
For the simulations, four test functions are proposed for the validity of the proposed combination technique and hence are proof-checked. Also, simulations are done to test the local adaptive mesh refinement. Finally the the problem at hand, the interpolation and the adaptive combination technique are tested against a predefined error function similar to adaptive grid generation methods. Lastly, an error analysis is performed with respect to threshold value for error and with maximum refinement level of trees. The implemented adaptive combination technique promising as the result as accuracy is similar or better as compared to the full grid method.