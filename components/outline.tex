\clearemptydoublepage

\phantomsection
\addcontentsline{toc}{chapter}{Outline of the Thesis}

\begin{center}
	\huge{Outline of the Thesis}
\end{center}




%--------------------------------------------------------------------
\section*{Part I: Introduction and Theory}

\noindent {\scshape Chapter 1: Introduction}\\
\noindent  This chapter presents a general overview of the thesis, what has been the reason and motivation behind this research, and its main objectives which are basically what we are trying to achieve. \\

\noindent {\scshape Chapter 2: Related Works} \\
\noindent  As any scientific work requires, this chapter presents all the related works and gives comprehensive literature review of published works. It can be seen how the research evolves to this point that now we try to push it further forward. All the material will be presented in the order of how progress has been done. \\

\noindent {\scshape Chapter 3: Methods}\\
\noindent  By methods in this chapter we mean the basic underlying principles and theories required to achieve the goal. All presented materials will be shown how they are related to the work at hand. Three primary methods presented are idea of sparse grid method and its relation to hierarchical basis, combination technique used to give another representation for sparse grid methods and finally data structure of trees specifically quadtrees in two-dimensional case. Finally some general important points will be discussed on what are the important factors at play in this kind of problem.\\



%--------------------------------------------------------------------
\section*{Part II: Implementation, Results and Conclusion}

\noindent {\scshape Chapter 4: My Implementation}\\
\noindent  This chapter presents the general idea of how combination technique is implemented in this case and more importantly how we are going to achieve the spatial adaptivity. Lastly, in this chapter different schemes are described.  \\

\noindent {\scshape Chapter 5: Results}\\
\noindent  Firstly in this chapter we verify the result of the main block used in the implementation and later we present the results of implementation presented under circumstances of different given schemes and effects of the two factors of adaptivity level and accuracy will be studied. \\

\noindent {\scshape Chapter 6: Conclusion} \\
\noindent  In similar to all scientific researches last chapter is dedicated to summary of what have been accomplished, what usage it can have and how to proceed from here for future related research and developments.\\
