\clearemptydoublepage

\phantomsection
\addcontentsline{toc}{chapter}{Outline of the Thesis}

\begin{center}
	\huge{Outline of the Thesis}
\end{center}




%--------------------------------------------------------------------
\section*{Part I: Introduction and Theory}

\noindent {\scshape Chapter 1: Introduction}\\
\noindent  This chapter presents a general overview of the thesis including the reason and motivation behind this research, and its main objectives. \\

\noindent {\scshape Chapter 2: Related Works} \\
\noindent  As any scientific work requires, this chapter presents all the related works and gives comprehensive literature review. It can be seen how the research evolved to this point. All the material will be presented in the order of how progress has been done. \\

\noindent {\scshape Chapter 3: Methods}\\
\noindent  By methods in this chapter basically means underlying principles and theories required to achieve the goal. All presented materials will be shown how they are related to the work at hand. Three primary methods presented are idea of sparse grid method and its relation to hierarchical basis, combination technique used to give another representation for sparse grid methods and finally data structure of trees specifically quadtrees in two-dimensional case. Finally some general important points related to the current thesis will be discussed.\\



%--------------------------------------------------------------------
\section*{Part II: Implementation, Results and Conclusion}

\noindent {\scshape Chapter 4: My Implementation}\\
\noindent  This chapter presents the general idea of combination technique including the spatial adaptivity and its implementation on this thesis. Lastly, in this chapter different schemes are described.  \\

\noindent {\scshape Chapter 5: Results}\\
\noindent  Firstly in this chapter we verify the result of the main block used in the implementation and later we present the results from different schemes and effects of the two factors of adaptivity level will be studied. \\

\noindent {\scshape Chapter 6: Conclusion} \\
\noindent  Similar to all scientific researches last chapter is dedicated to summary of what has been accomplished, what usage it can have and how to proceed from here for future related research and developments.\\
